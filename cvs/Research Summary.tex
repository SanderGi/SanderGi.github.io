\documentclass{article}
\usepackage[utf8]{inputenc}
\usepackage{amsmath}
\usepackage{amsfonts}
\usepackage{amssymb}
\usepackage{pgfplots}
\pgfplotsset{compat = newest}
\usepackage{graphicx}
\usepackage[shortlabels]{enumitem}
\usepackage[stable]{footmisc} % allow footnotes in section titles
\addtolength{\skip\footins}{2pc plus 5pt} % give footnotes some breathing room
\usepackage[T1]{fontenc} % fix underscore from \jobname
\usepackage{float}
\usepackage{subcaption} % subfigures

% Laplace Operator
\usepackage{mathtools}
\DeclarePairedDelimiterXPP\Laplace[1]{\operatorname{\mathcal{L}}}{\{}{\}}{}{#1}
\DeclarePairedDelimiterXPP\LaplaceInv[1]{\operatorname{\mathcal{L}^{-1}}}{\{}{\}}{}{#1}

% Derivatives
\newcommand*{\Partial}[2]{\frac{\partial #1}{\partial #2}}
\newcommand*{\Deriv}[2]{\frac{d #1}{d #2}}

% augmented coefficient matrix
\newenvironment{amatrix}[1]{%
    \left(\begin{array}{@{}*{#1}{c}|c@{}}
}{%
    \end{array}\right)
}

% rules 
\newcommand*{\vertbar}{\rule[-1ex]{0.5pt}{2.5ex}}
\newcommand*{\horzbar}{\rule[.5ex]{2.5ex}{0.5pt}}

% common sets
\newcommand*{\R}{\mathbb{R}}
\newcommand*{\C}{\mathbb{C}}
\newcommand*{\Z}{\mathbb{Z}}
\newcommand*{\N}{\mathbb{N}}
\newcommand*{\Q}{\mathbb{Q}}
\newcommand*{\D}{\mathbb{D}}

% inner products
\newcommand*{\Inner}[1]{\langle #1 \rangle}

% formatting
\newcommand*{\parspace}{\vspace{1em} \\}

% vectors
\newcommand*{\vx}{\mathbf{x}}
\newcommand*{\vy}{\mathbf{y}}
\newcommand*{\vz}{\mathbf{z}}
\newcommand*{\va}{\mathbf{a}}
\newcommand*{\vb}{\mathbf{b}}
\newcommand*{\vc}{\mathbf{c}}
\newcommand*{\vF}{\mathbf{F}}
\newcommand*{\vG}{\mathbf{G}}
\newcommand*{\vH}{\mathbf{H}}
\newcommand*{\vn}{\mathbf{n}}
\newcommand*{\vm}{\mathbf{m}}
\newcommand*{\vf}{\mathbf{f}}
\newcommand*{\vg}{\mathbf{g}}
\newcommand*{\vh}{\mathbf{h}}
\newcommand*{\vi}{{\mathbf{\hat{\textnormal{\bfseries\i}}}}}
\newcommand*{\vj}{{\mathbf{\hat{\textnormal{\bfseries\j}}}}}
\newcommand*{\vk}{{\mathbf{\hat{\textnormal{\bfseries k}}}}}
\newcommand*{\vzero}{\mathbf{0}}

\usepackage{xstring}
\title{\StrSubstitute{\jobname}{"}{}}
\author{Alexander Metzger}
\date{}

\begin{document}

\maketitle

\textbf{Introduction}. Information and Communications Technologies for Development
(ICT4D) is about making digital technologies accessible to everyone to promote social and
economic growth in developing countries. It is multidisciplinary and involves mobile
phones, computer networks, satellite systems, internet protocols, embedded systems, and
Human Computer Interaction. I take special interest in applying algorithm design, Machine
Learning, and mobile network technologies to aid environmental sustainability and
agricultural economic development.

\textbf{eKichabi v2}. Smallholder farmers in Sub-Saharan Africa lack accessible networking
platforms which makes them reliant on middlemen to sell their produce. Our team at the
ICTD Lab worked with collaborators to collect the largest agricultural phone directory to
date. A big challenge was making the digital directory accessible to users with limited
access to the internet or smartphones. To address this, I helped develop and maintain a
USSD (Unstructured Supplementary Service Data) application, which allowed users to
access the directory through basic mobile phones.

My specific role involved designing a custom binary protocol and compressi
scheme to optimize usage logging for the study, a crucial feature given the high cost of
mobile data in Tanzania. I took an active part in designing and improving the search fl
for farmers based on two user pilots. I also solved a potentially study breaking issue of
outdated data by developing a second USSD application to update information in our
directory. My main contribution was scaling our USSD application by more than 7000x.
This was a crucial step towards demonstrating the production readiness of the approach
and work towards turning it into a long-term government run project. Co-first-authori
the paper published at ACM SIGCHI 2024 with my mentor Ananditha Raghunath, I also
performed statistical tests and created plots. I was invited to present the findings at t
Para.chi DUB and the CHANGE seminar.

\textbf{Farmer.CHAT}. Smallholder farmers also lack access to trustworthy sources to teach
sustainable farming practices and resolve issues like crop disease. Continuing my research
with mobile technologies, I worked with Gooey.AI and our collaborators to develop and
deploy retrieval augmented systems to assist extension workers across Kenya, India,
Ethiopia, and Rwanda. The workflows are now used by 10s of millions of farmers and ha
increased the affordability and reach of extension efforts by 100x. This work was featured
by NVIDIA, OpenAI, and presented at the UN General Assembly.

The biggest challenge was handling low-resource languages where data is scarce
and foundation models lack relevant training. I organized efforts with partners to collect
domain specific datasets and train translation and language models. I came up with
custom glossary system to constrain models to specific meanings of non-English wor
based on context. I also pushed for support in regions without smartphone/WiFi access
and developed an Interactive Voice Response integration to allow phone users to call into a
verbal LLM conversation to retrieve their information.

\textbf{SAMIR}. The problem of environmentally sustainable decision-making applies not
just to farmers but to consumers of electronic products at large. Traditional life-cycle
assessments (LCAs) require domain experts and proprietary databases. Using a multimodal
information retrieval approach, we were able to create a set of algorithms to automatically
estimate LCAs within an average of 11\% of expert verified values leading to a user study
consensus that the tool is accurate and trustworthy.

I was actively involved in user study design and third-authored the paper submitted
to MobiSys 2025. I identified a lack of human-vetted but machine-parseab
environmental footprint datasets in literature. I thus laid the groundwork for our evaluation
and model training by parsing unstructured environment reports and online spec sheets
into an environmental footprint database of over 500 products. All radio-capable
technology is required to have an FCC report which would be a treasure trove of
information if they were not unstructured poorly scanned documents and unlabeled
images. To remedy this, I designed a computer vision algorithm to parse these into a
machine readable format. The approach is extensible and will enable future work to tap
into previously inaccessible data sources. It has the potential to be adopted by regulatory
bodies to make environmental impact information directly accessible to consumers.

\textbf{Conclusion}. A recurrent theme in my research is use of algorithm design and mobile
technology to offset gaps in information access for underserved languages/communities.
I'm continuing this work at Dr. Shwetak Patel's UbiComp Lab and with my own research
focused startup, Koel Labs. Funded by Mozilla and cloud credits, we are making language
learning accessible so non-native speakers can interact with information and communities
across language barriers.

\end{document}
